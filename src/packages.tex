\usepackage[T2A]{fontenc}
\usepackage[utf8]{inputenc}
\usepackage[russian]{babel}
\usepackage{bookmark}
\sloppy
\usepackage{hyperref}
\usepackage[dvips]{graphicx}
\usepackage{float}
\graphicspath{{picture/}}
\usepackage{amssymb,amsfonts,amsmath,amsthm}
\usepackage{wrapfig}
\usepackage{pgfplots}
\usepackage{geometry}
\geometry{left=2.5cm}
\geometry{right=1.5cm}
\geometry{top=2cm}
\geometry{bottom=2cm}
\usepackage{listings}
\usepackage{color}
\newcommand{\Mod}[1]

\usepackage{listings}
\usepackage{xcolor}

\lstset{ 
    language=C++,                     % Язык программирования
    basicstyle=\ttfamily\small,       % Основной стиль текста
    keywordstyle=\color{blue},        % Стиль ключевых слов
    commentstyle=\color{gray},        % Стиль комментариев
    stringstyle=\color{red},          % Стиль строк
    numberstyle=\tiny\color{gray},    % Стиль нумерации строк
    stepnumber=1,                     % Нумерация каждой строки
    numbersep=5pt,                    % Расстояние между номерами и кодом
    showspaces=false,                 % Показывать пробелы
    showstringspaces=false,           % Показывать пробелы в строках
    showtabs=false,                   % Показывать табуляции
    frame=single,                     % Рамка вокруг кода
    tabsize=4,                        % Размер табуляции
    captionpos=b,                     % Позиция заголовка
    breaklines=true,                  % Перенос длинных строк
    breakatwhitespace=false,          % Перенос строк только по пробелам
    title=\lstname                    % Показать имя файла
}