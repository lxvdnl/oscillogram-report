\section{Теоретическая часть программы}

\begin{quotation}
Программа написана на языке C++. Для графического интерфейса и отображения графика используется фреймворк qt.

Для построения осциллограммы движения на основе системы (\ref{eq:dimensionless system}) используется метод Рунге-Кутты четвертого порядка.
\end{quotation}

\subsection*{Описание метода Рунге-Кутты:}

\begin{quotation}
Метод Рунге-Кутты четвертого порядка предполагает вычисление новых значений функции в следующей точке через линейную комбинацию оценок (коэффициентов), полученных в нескольких промежуточных точках. Процесс расчета включает следующие шаги:

\begin{enumerate}
\item Выбор начальных условий $\tau_0, x_0, \dot x_0$
\item Вычисление промежуточных коэффициентов
\begin{quotation}
Разложим ДУ второго порядка на систему урфвнений первого порядка. Тогда $g(\tau)$ - функция производной $dx/d\tau$, \quad $f(\dot x)$ - функция производной $d^2x/d\tau^2$

Для каждого шага вычисляются четыре коэффициента \( k \) и \( l \), которые учитывают значения функции и её производных в разных точках внутри текущего шага:
\begin{align*}
k_1 &= \Delta \tau \cdot f(\dot x_n) \\
l_1 &= \Delta \tau \cdot g(\tau_n) \\
k_2 &= \Delta \tau \cdot f(\dot x_n + \frac{l_1}{2}) \\
l_2 &= \Delta \tau \cdot g(\tau_n + \frac{\Delta \tau }{2}) \\
k_3 &= \Delta \tau \cdot f(\dot x_n + \frac{l_2}{2}) \\
l_3 &= \Delta \tau \cdot g(\tau_n + \frac{\Delta \tau }{2}) \\
k_4 &= \Delta \tau \cdot f(\dot x_n + l_3) \\
l_4 &= \Delta \tau \cdot g(\tau_n + \Delta \tau)
\end{align*}

Здесь \(\Delta \tau\) - шаг интегрирования.
\end{quotation}

\item Обновление значений функции

\begin{quotation}
Новые значения \( x \) и \( \dot x \) вычисляются как средневзвешенные суммы промежуточных коэффициентов:
\begin{align*}
x_{n+1} &= x_n + \frac{1}{6} (k_1 + 2k_2 + 2k_3 + k_4) \\
\dot x_{n+1} &= \dot x_n + \frac{1}{6} (l_1 + 2l_2 + 2l_3 + l_4)
\end{align*}

Здесь \( x_n \) и \( \dot x_n \) - значения функции и её производной на текущем шаге, а \( x_{n+1} \) и \( \dot x_{n+1} \) - значения на следующем шаге.

Так же переходим к следующему шагу: $\tau_{n + 1} = \tau_n + \Delta \tau$ и повторяем шаги 2 и 3.
\end{quotation}
\end{enumerate}

В нашем случае нужно ещё учесть ударное взаимодействие с поверхностью $f(\tau)$ (\ref{eq:surface}). Поэтому на каждой итерации добавляем проверку $x = f(\tau)$ и если условие выполняется, домножаем \( \dot x \) на $-R$.
\end{quotation}